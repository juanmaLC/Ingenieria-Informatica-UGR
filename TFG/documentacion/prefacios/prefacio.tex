\chapter*{}
%\thispagestyle{empty}
%\cleardoublepage

%\thispagestyle{empty}

%\begin{titlepage}
 
 
\setlength{\centeroffset}{-0.5\oddsidemargin}
\addtolength{\centeroffset}{0.5\evensidemargin}
\thispagestyle{empty}

\noindent\hspace*{\centeroffset}\begin{minipage}{\textwidth}

\centering
\includegraphics[width=0.9\textwidth]{imagenes/logo_ugr.jpg}\\[1.4cm]

%\textsc{ \Large PROYECTO FIN DE CARRERA\\[0.2cm]}
%\textsc{ INGENIERÍA EN INFORMÁTICA}\\[1cm]
% Upper part of the page
% 

 \vspace{3.3cm}

%si el proyecto tiene logo poner aquí
%\includegraphics{imagenes/logo.png} 
 \vspace{0.5cm}

% Title

{\Huge\bfseries Diseño e integración de un sistema de clasificación de alteraciones del movimiento de pacientes con rodillas lesionadas\\
}
\noindent\rule[-1ex]{\textwidth}{3pt}\\[3.5ex]
{\large\bfseries  Detección de fatiga basado en entrenamientos de velocidad\\[4cm]}
\end{minipage}

\vspace{2.5cm}
\noindent\hspace*{\centeroffset}\begin{minipage}{\textwidth}
\centering

\textbf{Autor}\\ {Juan Manuel López Castro}\\[2.5ex]
\textbf{Directores}\\
{Oresti Baños Legrán \\
Miguel Damas Hermoso}\\[2cm]

%\includemovie{3cm}{3cm}{imagenes/tstc.gif}
\includegraphics[width=0.5\textwidth]{imagenes/atc.jpg}\\[0.1cm]
\textsc{Departamento de Arquitectura y Tecnología de Computadores}\\
\textsc{---}\\
Granada, Agosto de 2020
\end{minipage}
\addtolength{\textwidth}{\centeroffset}
\vspace{\stretch{2}}

 
\end{titlepage}






\cleardoublepage
\thispagestyle{empty}

\begin{center}
{\large\bfseries  Diseño e integración de un sistema de clasificación de alteraciones del movimiento de pacientes con rodillas lesionadas}\\
\end{center}
\begin{center}
Juan Manuel López Castro\\
\end{center}

%\vspace{0.7cm}
\noindent{\textbf{Palabras clave}: EMGs, fatiga, clasificación, lesión, diagnóstico, LCA, sistema nervioso central, aprendizaje automático, selección de características  }\\

\vspace{0.7cm}
\noindent{\textbf{Resumen}}\\

En la actualidad la electromiografía de superficie está jugando un papel muy importante apoyando a los fisioterapeutas y médicos para analizar el rendimiento de atletas y pacientes, así como para detectar posibles problemas musculares a la hora de realizar tanto una actividad física como una actividad cotidiana. 

Existen evidencias de que personas con lesiones en la rodilla, como por ejemplo, la rotura total o parcial del ligamento anterior cruzado, sufren cierta compensación muscular entre el cuadriceps y el isquiotibial. Basándonos en esto se plantea realizar un sistema automático para la detección de la fatiga muscular. Con dicho sistema un fisioterapeuta podría hallar que músculos entran en fatiga antes y determinar si hay o no cierta compensación entre músculos antagonistas, y por lo tanto probabilidad de mal funcionamiento de la rodilla debido a una lesión. 

Para ello se ha realizado un estudio sobre que tipos de clasificadores obtienen un mejor rendimiento a la hora de clasificar las señales de electromiografía entre fatiga y no fatiga. Se han hecho uso de datos de electromiografía tomados a un conjunto de bomberos que han realizado sentadillas, todos con un peso cargado del 70 por ciento de su RM. En cada repetición tomada también se ha registrado la velocidad de ejecución, para poder estimar la fatiga muscular.

\cleardoublepage


\thispagestyle{empty}


\begin{center}
{\large\bfseries Design and integration of a classification system for movement disorders in patients with injured knees}\\
\end{center}
\begin{center}
Juan Manuel López Castro\\
\end{center}

%\vspace{0.7cm}
\noindent{\textbf{Key words}: EMGs, fatigue, detection, injury, medical diagnostic, Central Nervous System, machine learning, feature selection  }\\

\vspace{0.7cm}
\noindent{\textbf{Abstract}}\\

Nowadays surface electromyography is playing a important role supporting to physiotherapist and medics to analyse athlete's performance as well as to detect possible muscular problems when they are performing some activity.

There is evidence that people with knee injuries, such as a total or partial tear of the anterior cruciate ligament, suffer some muscle compensation between the quadriceps and the hamstring. Based on this, the goal is create an automatic system for the detection of muscle fatigue. With this system, a physiotherapist could find that muscles go into fatigue earlier and determine whether or not there is some compensation between antagonist muscles and therefore the probability of knee malfunction due to injury.

To solve the problem, a study about which types of classifiers have better performance when classifying electromyography signals between fatigue and non-fatigue. Electromyography data was taken from a set of firefighters who have performed squats, all with a loaded weight of 70 percent of their MR. In each repetition taken, the execution speed has also been recorded, to be able to estimate muscle fatigue.





\chapter*{}
\thispagestyle{empty}

\noindent\rule[-1ex]{\textwidth}{2pt}\\[4.5ex]

Yo, \textbf{Juan Manuel López Castro}, alumno de la titulación INGENIERÍA INFORMÁTICA de la \textbf{Escuela Técnica Superior
de Ingenierías Informática y de Telecomunicación de la Universidad de Granada}, con DNI 77023144P, autorizo la
ubicación de la siguiente copia de mi Trabajo Fin de Grado en la biblioteca del centro para que pueda ser
consultada por las personas que lo deseen.

\vspace{6cm}

\noindent Fdo: Juan Manuel López Castro

\vspace{2cm}

\begin{flushright}
Granada a 2 de Septiembre de 2020
\end{flushright}


\chapter*{}
\thispagestyle{empty}

\noindent\rule[-1ex]{\textwidth}{2pt}\\[4.5ex]

D. \textbf{Oresti Baños Legrán }, Profesor del Área de Ingeniería de Sistemas y Automática del Departamento de Arquitectura y Tecnología de Computadores de la Universidad de Granada.

\vspace{0.3cm}

D. \textbf{Miguel Damas Hermoso}, Profesor del Área de Arquitectura y Tecnología de Computadores del Departamento de Arquitectura y Tecnología de Computadores de la Universidad de Granada.


\vspace{0.3cm}

\textbf{Informan:}

\vspace{0.3cm}

Que el presente trabajo, titulado \textit{\textbf{Diseño e integración de un sistema de clasificación de alteraciones del movimiento de pacientes con rodillas lesionadas}},
ha sido realizado bajo su supervisión por \textbf{Juan Manuel López Castro}, y autorizamos la defensa de dicho trabajo ante el tribunal
que corresponda.

\vspace{0.3cm}

Y para que conste, expiden y firman el presente informe en Granada a 2 de Septiembre de 2020

\vspace{0.3cm}

\textbf{Los directores:}

\vspace{5cm}

\noindent \textbf{Oresti Baños Legrán \ \ \ \ \ Miguel Damas Hermoso}

\chapter*{Agradecimientos}
\thispagestyle{empty}

       \vspace{1cm}


Gracias por el apoyo a todos mis familiares y amigos.

