\chapter{Especificación de requisitos}
En este capítulo se tratarán tanto los requisitos funciones como los no funcionales del proyecto. Con esto se obtendrá una idea general sobre la funcionalidad final que se quiere obtener.

\section{Requisitos funcionales}
Como bien es sabido, los requisitos funcionales definen la funcionalidad de nuestro sistema software. Los requisitos funcionales de este proyecto son:
\begin{itemize}
\item Captar las señales de electromiografía durante la actividad física en cuestión, en este caso, sentadillas.
\item Filtrar las señales obtenidas con el fin de obtener el menor ruido posible.
\item Analizar la activación muscular de cada señal, para determinar el tamaño de ventana de cada repetición.
\item Obtener las características más representativas de la señal.
\item Analizar la velocidad registrada y asignar a cada repetición la correspondiente fatiga.
\item Seleccionar las características más representativas de la fatiga.
\item Crear un dataset con las características para la clasificación.
\item Normalizar el dataset de las características, para así aumentar el rendimiento del software.
\item Clasificación de las señales entre fatiga y no fatiga.
\item Aviso visual y sonoro en caso de detectar dicha fatiga.


\end{itemize}

\section{Requisitos no funcionales}
Con respecto a los requisitos no funcionales, son los encargados de dictar las condiciones básicas que debe de seguir nuestro software, es decir, las características generales. Los requisitos no funcionales serían los siguientes:
\begin{itemize}
\item Sistema intuitivo de usar y cómodo a la hora de llevar puesto para la realización de la actividad física.
\item Conexión a dispositivo Android que muestre los resultados en tiempo real.
\item Rendimiento óptimo, para si obtener la menor latencia posible.
\item Buen electromiógrafo, para un mayor rendimiento.
\end{itemize}